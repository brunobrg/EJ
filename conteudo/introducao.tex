\subsection{Empresa Júnior, o que é?} % (fold)
\label{sub:oQueE}

Empresa Júnior - EJ é uma associação civil, ou seja, com um objetivo comum e bem definido formada exclusivamente por alunos da graduação e sem fins lucrativos. A receita oriunda dos projetos da EJ deve ser reinvestida na própria EJ e não pode ser distribuída entre seus membros, porêm, a remuneração existe e é incentivada, seja na forma de viagem para congressos e/ou palestras e cursos.

A EJ deve ser ainda autônoma, respondendo por todos os seus atos. Para tal, a EJ não deve sofrer intervenção externa na sua gestão, porêm é de suma importância a participação de um orientador no acompanhamento da produção intelectual, sobretudo na confecção dos projetos.


\subsection{Objetivos de uma Empresa Júnior} % (fold)
\label{sub:objetivos}

\begin{itemize}
	\item Uma Empresa Júnior pripicia o desenvolvimento técnico e interpessoal dos alunos, tais como capacidade de gerenciamento, oratória, liderança, empreendedorismo, proatividade, entre outos.\er.
	\item Possibilidade de colocar o conteúdo teórico em prática.\er.
	\item Elo entre sociedade e o acadêmico. Este fato complementa a missão da Universidade de Brasília, que é de desenvolver benefícios para o ambiente na qual está inserida.\er.
	\item Dar publicidade à Universidade de Brasília.
	\item Prática de responsabilidade social corporativa, através do volutanriado.
	\item Oferecer 	projetos a custos inferiores com qualidade. Isto possibilita, também, um desenvolvimento econômico local, graças à consultoria acessível para micro e pequenos empresários.
	\item Ranking do MEC, as IES que adotam empresa júnior na sua estrutura são mais bem cotadas na pesquisa.
\end{itemize}

